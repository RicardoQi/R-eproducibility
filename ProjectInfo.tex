% Options for packages loaded elsewhere
\PassOptionsToPackage{unicode}{hyperref}
\PassOptionsToPackage{hyphens}{url}
%
\documentclass[
]{article}
\title{R-eproducibility}
\author{RicardoQi}
\date{January 25th, 2022}

\usepackage{amsmath,amssymb}
\usepackage{lmodern}
\usepackage{iftex}
\ifPDFTeX
  \usepackage[T1]{fontenc}
  \usepackage[utf8]{inputenc}
  \usepackage{textcomp} % provide euro and other symbols
\else % if luatex or xetex
  \usepackage{unicode-math}
  \defaultfontfeatures{Scale=MatchLowercase}
  \defaultfontfeatures[\rmfamily]{Ligatures=TeX,Scale=1}
\fi
% Use upquote if available, for straight quotes in verbatim environments
\IfFileExists{upquote.sty}{\usepackage{upquote}}{}
\IfFileExists{microtype.sty}{% use microtype if available
  \usepackage[]{microtype}
  \UseMicrotypeSet[protrusion]{basicmath} % disable protrusion for tt fonts
}{}
\makeatletter
\@ifundefined{KOMAClassName}{% if non-KOMA class
  \IfFileExists{parskip.sty}{%
    \usepackage{parskip}
  }{% else
    \setlength{\parindent}{0pt}
    \setlength{\parskip}{6pt plus 2pt minus 1pt}}
}{% if KOMA class
  \KOMAoptions{parskip=half}}
\makeatother
\usepackage{xcolor}
\IfFileExists{xurl.sty}{\usepackage{xurl}}{} % add URL line breaks if available
\IfFileExists{bookmark.sty}{\usepackage{bookmark}}{\usepackage{hyperref}}
\hypersetup{
  pdftitle={R-eproducibility},
  pdfauthor={RicardoQi},
  hidelinks,
  pdfcreator={LaTeX via pandoc}}
\urlstyle{same} % disable monospaced font for URLs
\usepackage[margin=1in]{geometry}
\usepackage{color}
\usepackage{fancyvrb}
\newcommand{\VerbBar}{|}
\newcommand{\VERB}{\Verb[commandchars=\\\{\}]}
\DefineVerbatimEnvironment{Highlighting}{Verbatim}{commandchars=\\\{\}}
% Add ',fontsize=\small' for more characters per line
\usepackage{framed}
\definecolor{shadecolor}{RGB}{248,248,248}
\newenvironment{Shaded}{\begin{snugshade}}{\end{snugshade}}
\newcommand{\AlertTok}[1]{\textcolor[rgb]{0.94,0.16,0.16}{#1}}
\newcommand{\AnnotationTok}[1]{\textcolor[rgb]{0.56,0.35,0.01}{\textbf{\textit{#1}}}}
\newcommand{\AttributeTok}[1]{\textcolor[rgb]{0.77,0.63,0.00}{#1}}
\newcommand{\BaseNTok}[1]{\textcolor[rgb]{0.00,0.00,0.81}{#1}}
\newcommand{\BuiltInTok}[1]{#1}
\newcommand{\CharTok}[1]{\textcolor[rgb]{0.31,0.60,0.02}{#1}}
\newcommand{\CommentTok}[1]{\textcolor[rgb]{0.56,0.35,0.01}{\textit{#1}}}
\newcommand{\CommentVarTok}[1]{\textcolor[rgb]{0.56,0.35,0.01}{\textbf{\textit{#1}}}}
\newcommand{\ConstantTok}[1]{\textcolor[rgb]{0.00,0.00,0.00}{#1}}
\newcommand{\ControlFlowTok}[1]{\textcolor[rgb]{0.13,0.29,0.53}{\textbf{#1}}}
\newcommand{\DataTypeTok}[1]{\textcolor[rgb]{0.13,0.29,0.53}{#1}}
\newcommand{\DecValTok}[1]{\textcolor[rgb]{0.00,0.00,0.81}{#1}}
\newcommand{\DocumentationTok}[1]{\textcolor[rgb]{0.56,0.35,0.01}{\textbf{\textit{#1}}}}
\newcommand{\ErrorTok}[1]{\textcolor[rgb]{0.64,0.00,0.00}{\textbf{#1}}}
\newcommand{\ExtensionTok}[1]{#1}
\newcommand{\FloatTok}[1]{\textcolor[rgb]{0.00,0.00,0.81}{#1}}
\newcommand{\FunctionTok}[1]{\textcolor[rgb]{0.00,0.00,0.00}{#1}}
\newcommand{\ImportTok}[1]{#1}
\newcommand{\InformationTok}[1]{\textcolor[rgb]{0.56,0.35,0.01}{\textbf{\textit{#1}}}}
\newcommand{\KeywordTok}[1]{\textcolor[rgb]{0.13,0.29,0.53}{\textbf{#1}}}
\newcommand{\NormalTok}[1]{#1}
\newcommand{\OperatorTok}[1]{\textcolor[rgb]{0.81,0.36,0.00}{\textbf{#1}}}
\newcommand{\OtherTok}[1]{\textcolor[rgb]{0.56,0.35,0.01}{#1}}
\newcommand{\PreprocessorTok}[1]{\textcolor[rgb]{0.56,0.35,0.01}{\textit{#1}}}
\newcommand{\RegionMarkerTok}[1]{#1}
\newcommand{\SpecialCharTok}[1]{\textcolor[rgb]{0.00,0.00,0.00}{#1}}
\newcommand{\SpecialStringTok}[1]{\textcolor[rgb]{0.31,0.60,0.02}{#1}}
\newcommand{\StringTok}[1]{\textcolor[rgb]{0.31,0.60,0.02}{#1}}
\newcommand{\VariableTok}[1]{\textcolor[rgb]{0.00,0.00,0.00}{#1}}
\newcommand{\VerbatimStringTok}[1]{\textcolor[rgb]{0.31,0.60,0.02}{#1}}
\newcommand{\WarningTok}[1]{\textcolor[rgb]{0.56,0.35,0.01}{\textbf{\textit{#1}}}}
\usepackage{graphicx}
\makeatletter
\def\maxwidth{\ifdim\Gin@nat@width>\linewidth\linewidth\else\Gin@nat@width\fi}
\def\maxheight{\ifdim\Gin@nat@height>\textheight\textheight\else\Gin@nat@height\fi}
\makeatother
% Scale images if necessary, so that they will not overflow the page
% margins by default, and it is still possible to overwrite the defaults
% using explicit options in \includegraphics[width, height, ...]{}
\setkeys{Gin}{width=\maxwidth,height=\maxheight,keepaspectratio}
% Set default figure placement to htbp
\makeatletter
\def\fps@figure{htbp}
\makeatother
\setlength{\emergencystretch}{3em} % prevent overfull lines
\providecommand{\tightlist}{%
  \setlength{\itemsep}{0pt}\setlength{\parskip}{0pt}}
\setcounter{secnumdepth}{-\maxdimen} % remove section numbering
\ifLuaTeX
  \usepackage{selnolig}  % disable illegal ligatures
\fi

\begin{document}
\maketitle

\hypertarget{project-info}{%
\section{Project Info}\label{project-info}}

This project is written on January 25th, 2022. The project is cloned to
\href{https://github.com/RicardoQi/R-eproducibility.git}{RicardoQi
Github homepage}

\hypertarget{working-report-in-steps}{%
\section{Working Report in Steps}\label{working-report-in-steps}}

The csv file is loaded using the following pathway and named as Mydata
in the data base.

\begin{Shaded}
\begin{Highlighting}[]
\NormalTok{Mydata }\OtherTok{\textless{}{-}} \FunctionTok{read.csv}\NormalTok{(}\StringTok{"C:}\SpecialCharTok{\textbackslash{}\textbackslash{}}\StringTok{Users}\SpecialCharTok{\textbackslash{}\textbackslash{}}\StringTok{qi199}\SpecialCharTok{\textbackslash{}\textbackslash{}}\StringTok{Documents}\SpecialCharTok{\textbackslash{}\textbackslash{}}\StringTok{R{-}eproducibility}\SpecialCharTok{\textbackslash{}\textbackslash{}}\StringTok{InputData}\SpecialCharTok{\textbackslash{}\textbackslash{}}\StringTok{FallopiaData.csv"}\NormalTok{)}
\end{Highlighting}
\end{Shaded}

\begin{center}\rule{0.5\linewidth}{0.5pt}\end{center}

Then, rows with ``Total'' values less than 60 are removed.

\begin{Shaded}
\begin{Highlighting}[]
\NormalTok{Mydata1 }\OtherTok{\textless{}{-}} \FunctionTok{subset}\NormalTok{(Mydata, Total}\SpecialCharTok{\textgreater{}}\DecValTok{60}\NormalTok{)}
\end{Highlighting}
\end{Shaded}

\begin{center}\rule{0.5\linewidth}{0.5pt}\end{center}

Mydata is rearranged and the order of columns are altered, and only four
of the columns are shown.

\begin{Shaded}
\begin{Highlighting}[]
\NormalTok{Mydata2 }\OtherTok{\textless{}{-}} \FunctionTok{subset}\NormalTok{(Mydata1, }\AttributeTok{select =} \FunctionTok{c}\NormalTok{(Total, Taxon, Scenario, Nutrients))}
\end{Highlighting}
\end{Shaded}

\begin{center}\rule{0.5\linewidth}{0.5pt}\end{center}

A new column ``TotalG'' is created to replace the ``Total'' column with
gram becomes the measuring unit instead of mg.

\begin{Shaded}
\begin{Highlighting}[]
\NormalTok{Mydata2}\SpecialCharTok{$}\NormalTok{TotalG }\OtherTok{\textless{}{-}}\NormalTok{ Mydata2}\SpecialCharTok{$}\NormalTok{Total }\SpecialCharTok{/} \DecValTok{1000} 
\NormalTok{Mydata2 }\OtherTok{\textless{}{-}}\NormalTok{ Mydata2[,}\FunctionTok{c}\NormalTok{(}\DecValTok{5}\NormalTok{,}\DecValTok{2}\SpecialCharTok{:}\DecValTok{4}\NormalTok{)] }
\end{Highlighting}
\end{Shaded}

\begin{center}\rule{0.5\linewidth}{0.5pt}\end{center}

The following function is written to do simple calculations based on
data and operation prompted by the user.

The function is named ctm(). It asks for entering two inputs from the
user. The first one is considered vector and converted to numeric type
for further calculations, and the second variable is a string that help
determine the operation later.

\begin{Shaded}
\begin{Highlighting}[]
\NormalTok{ctm }\OtherTok{\textless{}{-}} \ControlFlowTok{function}\NormalTok{() \{}
\NormalTok{  var1 }\OtherTok{\textless{}{-}} \FunctionTok{readline}\NormalTok{(}\StringTok{"Enter the sequence of numbers separated by comma : "}\NormalTok{)}
\NormalTok{  var2 }\OtherTok{\textless{}{-}} \FunctionTok{readline}\NormalTok{(}\StringTok{"Enter the operation : "}\NormalTok{)}
\NormalTok{  var1 }\OtherTok{\textless{}{-}} \FunctionTok{strsplit}\NormalTok{(var1,}\StringTok{\textquotesingle{},\textquotesingle{}}\NormalTok{)}
\NormalTok{  var1 }\OtherTok{\textless{}{-}} \FunctionTok{as.numeric}\NormalTok{(}\FunctionTok{unlist}\NormalTok{(var1))}
  
\NormalTok{  avg }\OtherTok{\textless{}{-}} \FunctionTok{mean}\NormalTok{(var1)}
\NormalTok{  ttl }\OtherTok{\textless{}{-}} \FunctionTok{sum}\NormalTok{(var1)}
\NormalTok{  cnt }\OtherTok{\textless{}{-}} \FunctionTok{table}\NormalTok{(var1)}
  
  \ControlFlowTok{if}\NormalTok{ (var2 }\SpecialCharTok{==} \StringTok{"Average"}\NormalTok{) \{}
    \FunctionTok{print}\NormalTok{(avg)}
\NormalTok{  \} }\ControlFlowTok{else} \ControlFlowTok{if}\NormalTok{ (var2 }\SpecialCharTok{==} \StringTok{"Sum"}\NormalTok{) \{}
    \FunctionTok{print}\NormalTok{(ttl)}
\NormalTok{  \} }\ControlFlowTok{else} \ControlFlowTok{if}\NormalTok{ (var2 }\SpecialCharTok{==} \StringTok{"Observations"}\NormalTok{) \{}
    \FunctionTok{print}\NormalTok{(cnt)}
\NormalTok{  \} }\ControlFlowTok{else}\NormalTok{ \{}
    \FunctionTok{print}\NormalTok{(}\StringTok{"Error: Wrong operation!"}\NormalTok{)}
\NormalTok{  \}}
  
\NormalTok{\}}
\end{Highlighting}
\end{Shaded}

avg, ttl, and cnt are three options that the user can choose to control
their data in vector \#1. If the user enters other content, the function
would give error.

\begin{center}\rule{0.5\linewidth}{0.5pt}\end{center}

The total number of observations in the `Taxon' column is calculated
using dplyr.

\begin{Shaded}
\begin{Highlighting}[]
\NormalTok{num\_obs }\OtherTok{\textless{}{-}} \ControlFlowTok{function}\NormalTok{()\{}
  \FunctionTok{library}\NormalTok{(}\StringTok{"dplyr"}\NormalTok{)}
\NormalTok{  Mydata2 }\SpecialCharTok{\%\textgreater{}\%} \FunctionTok{count}\NormalTok{(Taxon)}
\NormalTok{\}}
\FunctionTok{num\_obs}\NormalTok{()}
\end{Highlighting}
\end{Shaded}

\begin{verbatim}
## 
## 载入程辑包:'dplyr'
\end{verbatim}

\begin{verbatim}
## The following objects are masked from 'package:stats':
## 
##     filter, lag
\end{verbatim}

\begin{verbatim}
## The following objects are masked from 'package:base':
## 
##     intersect, setdiff, setequal, union
\end{verbatim}

\begin{verbatim}
##   Taxon  n
## 1 bohem 22
## 2 japon 23
\end{verbatim}

\begin{center}\rule{0.5\linewidth}{0.5pt}\end{center}

The average ``TotalG'' for each of the two Nutrient concentrations is
calculated using the aggregate() function.

\begin{Shaded}
\begin{Highlighting}[]
\NormalTok{avg\_nut }\OtherTok{\textless{}{-}} \ControlFlowTok{function}\NormalTok{()\{}
  \FunctionTok{aggregate}\NormalTok{(Mydata2[, }\DecValTok{1}\NormalTok{], }\FunctionTok{list}\NormalTok{(Mydata2}\SpecialCharTok{$}\NormalTok{Nutrients), mean)}
\NormalTok{\}}
\FunctionTok{avg\_nut}\NormalTok{()}
\end{Highlighting}
\end{Shaded}

\begin{verbatim}
##   Group.1          x
## 1    high 0.06646674
## 2     low 0.06407000
\end{verbatim}

\begin{center}\rule{0.5\linewidth}{0.5pt}\end{center}

Finally, the newly synthesized data is saved to a new csv file named
WrangledData.csv.

\begin{Shaded}
\begin{Highlighting}[]
\FunctionTok{write.csv}\NormalTok{(Mydata2, }\StringTok{"C:}\SpecialCharTok{\textbackslash{}\textbackslash{}}\StringTok{Users}\SpecialCharTok{\textbackslash{}\textbackslash{}}\StringTok{qi199}\SpecialCharTok{\textbackslash{}\textbackslash{}}\StringTok{Documents}\SpecialCharTok{\textbackslash{}\textbackslash{}}\StringTok{R{-}eproducibility}\SpecialCharTok{\textbackslash{}\textbackslash{}}\StringTok{Output}\SpecialCharTok{\textbackslash{}\textbackslash{}}\StringTok{WrangledData.csv"}\NormalTok{, }\AttributeTok{row.names =} \ConstantTok{FALSE}\NormalTok{)}
\end{Highlighting}
\end{Shaded}


\end{document}
